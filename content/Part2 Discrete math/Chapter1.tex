\chapterimage{orange2.jpg}
\chapterspaceabove{6.75cm} 
\chapterspacebelow{7.25cm} 
\chapter{Boolean Algebra}
    In the first chapter of this book, we discussed the very basic of mathematics, 
    proof and propositions. This chapter aims to excavate mathematical logics in 
    further details. With Boolean Algebra, we can explain more thoroughly on
    the mechanism of logics and mathematical proof, and on top of that, we can 
    figure out how computer functions, as well as how integrated circuits are 
    constructed.
    \section{Boolean Expression and Truth Table}
        We call Boolean Algebra "Algebra" of course, because it possesses property of Algebra.
        We will start with basic algebra operation and thus, proceed to get the rule of 
        Boolean operation, which we call the "Truth Table".
    \subsection{Property of Algebra Operation}
        For normal algebra operation of numbers, we have the following general law
        by representing the number in $x$, $y$, and $z$.
        \begin{table}[ht]
            \label{algelaw}
            \centering
            \caption{Common Algebraic Laws}
            \begin{tabular}{lll}
            \toprule
            \textbf{Law} & \textbf{Addition Expression} & \textbf{Multiplication Expression} \\
            \midrule
            Identity & \( x + 0 = x \) & \( x \cdot 1 = x \) \\
            Property of Zero & & \( x \cdot 0 = 0 \) \\
            Inverse & \( x + (-x) = 0 \) & \( x \cdot x^{-1} = 1 \), for \( x \neq 0 \) \\
            Commutative & \( x + y = y + x \) & \( x \cdot y = y \cdot x \) \\
            Associative & \( (x + y) + z = x + (y + z) \) & \( (x \cdot y) \cdot z = x \cdot (y \cdot z) \) \\
            Distributive & &\( x \cdot (y + z) = x \cdot y + x \cdot z \) \\
            \bottomrule
            \end{tabular}
        \end{table}
        In this case, we call $x$, $y$, and $z$ variables as in programming. for 
        all possible values of these variables, these laws hold.
        Why these laws are important is that all algebra expressions is derived
        from these laws, which means we can prove any equality that is correct.
        We take $-1 \time -1$ as an example. One may say without a second thought that
        the answer is 1. But why? Can we prove it? Now we try to prove it using the
        common algebraic laws.
        \begin{example}
            Prove that $(-1)\times (-1) = 1$
        \end{example}
        \begin{proof}
            \begin{align*}
                 & ( -1) \times ( -1)\\
                = & (( -1) \times ( -1)) +0\ ( Addition\ Identity\ )\\
                = & (( -1) \times ( -1)) +\ (( -1) +1) \ ( Addition\ Inverse)\\
                = & ((( -1) \times ( -1) +( -1)) +1\ ( Associative\ Law\ of\ Addition)\\
                = & ((( -1) \times ( -1) +(( -1) \times 1)) +1\ ( Multiplication\ Identity)\\
                = & (( -1) \times (( -1) +1)) +1\ ( Distributive\ Law)\\
                = & (( -1) \times 0) \ +\ 1\ ( Addition\ Inverse)\\
                = & 0+1\ ( Multiplication\ Property\ of\ Zero)\\
                = & 1+0\ (Commutative\ Law\ of\ Addition)\\
                = & 1\ ( Addition\ Identity)
            \end{align*}
            \begin{remark}
                Some people may not understand why we have to write $0+1$ into $1+0$. This is because
                Addition Identity is only defined in the table as $x+0=x$, so we need to apply Commutative
                law of addition to fit it into the known conclusion.
            \end{remark}
        \end{proof}

        After seeing this, you may think, do we really have to know this to make sure that $-1\times-1=1$?
        Of course not, but keep in mind that \textbf{Every mathematical conclusion cannot be simply referred as
        rules such as "a negative times negative give you a positive number", but by meticulous, reasonable, and 
        replicable proof.}

    \subsection{Boolean Expression and Truth Table}
        Hold on a second, isn't this chapter on Boolean Algebra? Why we still need to go over these
        old knowledge from primary school? Well, this is because we will prove Boolean expression in the
        same way. Before that, we introduce Boolean value and operations. 
        % Definition of Boolean Values
        \begin{definition}[Boolean Values]
        In mathematics and computer science, a Boolean value is defined as an element of the Boolean domain \(B\), which can be mathematically represented as:
        \[
        B = \{0, 1\}
        \]
        where:
        \begin{itemize}
            \item \(0\) typically represents \textit{false}
            \item \(1\) typically represents \textit{true}
        \end{itemize}
        \end{definition}
        Some may be confused by true and false here. Just recall what we have done to propositions in
        the first chapter in this book. When a statement holds for given condition, we take it as
        correct, while incorrect when it does not hold. Boolean value is the basis of Boolean Expression,
        and \textbf{all valid boolean expression could be reduced or simplified to a Boolean value}.
        % Boolean Operations
        \begin{definition}[Boolean Operations]
        Boolean algebra involves operations such as AND, OR, NOT, XOR, which operate on these Boolean values. These operations are defined as follows:
        \begin{itemize}
            \item \textbf{AND} (\(\land\)): An operation on two Boolean values that returns \textit{true} if both operands are \textit{true}, otherwise returns \textit{false}.
            \item \textbf{OR} (\(\lor\)): An operation on two Boolean values that returns \textit{true} if at least one of the operands is \textit{true}, otherwise returns \textit{false}.
            \item \textbf{NOT} (\(\lnot\)): A unary operation that returns \textit{true} if the operand is \textit{false} and vice versa.
            \item \textbf{XOR} (\(\oplus\)): An operation on two Boolean values that returns \textit{true} if the operands are different, otherwise returns \textit{false}.
        \end{itemize}
        \begin{remark}
            There are more Boolean operators to be discussed later.
        \end{remark}
        \end{definition}
        In computer science, Boolean values are fundamental in conditional statements and loops, where they determine the flow of control in algorithms and programs. They are also essential in the design of electronic circuits and digital computing.
        The following table shows the truth table of these basic Boolean operations. the first column
        shows the combination of inputs for the specific operator, and the second column shows the
        result of operation.
        \begin{table}[ht] \label{truthtab}
            \centering % This centers the entire table construct
            
            \noindent
            \begin{tabular}{c|c}
            \textbf{A} & $\neg \textbf{A}$ \\
            \hline
            F & T \\
            T & F \\
            \end{tabular}
            \quad % Reduce the space between tables
            \begin{tabular}{cc|c}
            \textbf{A} & \textbf{B} & $\textbf{A} \land \textbf{B}$ \\
            \hline
            F & F & F \\
            F & T & F \\
            T & F & F \\
            T & T & T \\
            \end{tabular}
            
            \vspace{2mm} % Reduce the vertical space between the rows of tables
            
            \noindent
            \begin{tabular}{cc|c}
            \textbf{A} & \textbf{B} & $\textbf{A} \lor \textbf{B}$ \\
            \hline
            F & F & F \\
            F & T & T \\
            T & F & T \\
            T & T & T \\
            \end{tabular}
            \quad % Reduce the space between tables
            \begin{tabular}{cc|c}
            \textbf{A} & \textbf{B} & $\textbf{A} \oplus \textbf{B}$ \\
            \hline
            F & F & F \\
            F & T & T \\
            T & F & T \\
            T & T & F \\
            \end{tabular}
            
            \caption{Common Boolean Operators Truth Tables}
        \end{table}

        It's noticeable that different Boolean operators may differ in the number of input. $\lnot$ operator
        takes only one Boolean variable (input) to get an output by negating the input, however the rest 
        take two inputs and produce one output. 
        
        At the beginning of the section, we have defined Boolean
        space $B=\{0,1\}$, but why George Boole(The British mathematician who initiate this idea) define 
        Boolean value with a set containing only zero and one? This is actually because, \textbf{The foundation
        of Boolean Algebra is set upon operations involving 0 and 1}. To explicit, I need to clarify that
        among the four operators introduced above, $\lnot$, $\land$, and $\lor$ are the very basic of all
        Boolean expressions, which means other Boolean operators could be written in an equivalent form with
        these basic operators. For example, we can write XOR($\oplus$) in the other three operators:
        \begin{equation}
            \label{xor}
            A \oplus B= (A \land \lnot B) \lor(\lnot A \land B)
        \end{equation}
        It doesn't really matter if you find the RHS strange,as all we need now is just know the fact 
        that we can write it in this form. We will discuss the expression in details in the next section.
        Now let's refocus on Boolean space \{0,1\}. We all know that there are four basic operations in 
        algebra which are addition, subtraction, multiplication, and division. But generally, we say there 
        are only two basic operations in algebra, which are $+$ and $\times$, as subtraction and division are
        just inverse operation of addition and multiplication. Let's look in to the table of addition and 
        multiplication of 0 and 1 listed below.
        \begin{table}[ht]
            \centering % This centers the entire table construct
            \noindent
            \begin{tabular}{cc|c}
                A & B & $A \times B$ \\
                \hline
                0 & 0 & 0 \\
                0 & 1 & 0 \\
                1 & 0 & 0 \\
                1 & 1 & 1 \\
            \end{tabular}
            \quad
            \begin{tabular}{cc|c}
                A & B & $A + B$ \\
                \hline
                0 & 0 & 0 \\
                0 & 1 & 1 \\
                1 & 0 & 1 \\
                1 & 1 & 1 \\     
            \end{tabular}
        \caption{Addition and Multiplication Rule of 0 and 1}
        \end{table}

        Now you may have realized that this is exactly the truth table for $\land$ and $\lor$, you may
        check table \ref{truthtab}.
    
        \subsection{Boolean Identities}
        Now recall that $A \oplus B= (A \land \lnot B) \lor(\lnot A \land B)$. In this
        expression, we cannot get a direct answer from the RHS by using the truth table of $\lor$. So,
        we need to try harder to get the truth table for the expression as shown in the table below.
        \begin{table}[ht]
            \centering
            \caption{Truth table for \((A \land \neg B) \lor (\neg A \land B)\)}
            \begin{tabular}{cc|c|c|c|c|c}
            \hline
            \( A \) & \( B \) & \( \neg B \) & \( A \land \neg B \) & \( \neg A \) & \( \neg A \land B \) & \( (A \land \neg B) \lor (\neg A \land B) \) \\ \hline
            F & F & T & F & T & F & F \\
            F & T & F & F & T & T & T \\
            T & F & T & T & F & F & T \\
            T & T & F & F & F & F & F \\ \hline
            \end{tabular}
            
        \end{table}

        With this method, we can find truth table for more complex expressions, and some of them
        are categorized as \textbf{Basic Boolean Identities}. The LHS and RHS are could be proven qaual
        by listing truth table respectively.

        \begin{table}[H]
            \centering
            \caption{Basic Boolean Identities}
            \begin{tabular}{lll}
            \toprule
            \textbf{Identity} & \textbf{AND Form} & \textbf{OR Form} \\
            \midrule
            Idempotent Law & \( x \cdot x = x \) & \( x + x = x \) \\
            Identity Law & \( x \cdot 1 = x \) & \( x + 0 = x \) \\
            Domination Law & \( x \cdot 0 = 0 \) & \( x + 1 = 1 \) \\
            Complement Law & \( x \cdot \lnot x = 0 \) & \( x + \lnot x = 1 \) \\
            Double Negation Law & \( \lnot(\lnot x) = x \) & \( \lnot(\lnot x) = x \)\\
            Commutative Law & \( x \cdot y = y \cdot x \) & \( x + y = y + x \) \\
            Associative Law & \( x \cdot (y \cdot z) = (x \cdot y) \cdot z \) & \( x + (y + z) = (x + y) + z \) \\
            Distributive Law & \( x \cdot (y + z) = (x \cdot y) + (x \cdot z) \) & \( x + (y \cdot z) = (x + y) \cdot (x + z) \) \\
            De Morgan's Law & \( \lnot(x + y) = \lnot x \cdot \lnot y \) & \( \lnot(x \cdot y) = \lnot x + \lnot y \) \\
            Absorption Law & \( x \cdot (x + y) = x \) & \( x + (x \cdot y) = x \) \\
            \bottomrule
            \end{tabular}
        \end{table}
        \begin{remark}
            If you are confused by the algebra form of Boolean identities, just take $x$, $y$, $z$ as
            $A$, $B$, and $C$; take 0/1 as F/T; take + as $\lor$, and $\times$ as $\land$.
        \end{remark}
        It is quite important to be familiar with these rules, as they are just as essential as
        the normal algebra laws we've learned before, since we may simplify complex Boolean expressions
        using these rules. Also, the proof of some of these identities using truth table will be 
        added to the problem set for this section. If you remain any doubt or confusion about any laws
        given, just try to get the truth table for the LHS and RHS of the identity, simple as that.

        This is actually not yet a complete table of Boolean Identities, two laws are still missing
        from the table, because they are relevant to secondary operator, one of which is already mentioned($\oplus$).

        With these basic Boolean laws, we can derive more interesting and useful theorems. A commonly used theorem when simplifying
        Boolean expression is concensus (or also called redundancy) theorem.
        \begin{theorem}[Concensus Theorem]
            The Consensus Theorem helps simplifying Boolean expressions by eliminating a redundant term, and like other laws, it has both
            OR and AND form. 
            It states that for any Boolean variables $x$, $y$, and $z$:
            \begin{equation*}
                xy\lor\bar{x}z\lor yz=xy\lor\bar{x}z
            \end{equation*}
            Which is equivalent to
            \begin{equation*}
                (x\lor y)(\bar{x}\lor z)(y\lor z)=(x\lor y)(\bar{x}\lor z)
            \end{equation*}
        \end{theorem}
        \begin{proof}
            $$\begin{aligned}
                xy\lor\bar{x}z\lor yz& =xy\lor\bar{x}z\lor(x\lor\bar{x})yz  \\
                &=xy\lor\bar{x}z\lor xyz\lor\bar{x}yz \\
                &=(xy\lor xyz)\lor(\bar{x}z\lor\bar{x}yz) \\
                &=xy(1\lor z)\lor\bar{x}z(1\lor y) \\
                &=xy\lor\bar{x}z
                \end{aligned}$$
            Or in another notation.
            $$\begin{aligned}
                xy + \overline{x}z + yz &= xy + \overline{x}z + (x + \overline{x})yz \\
                &= xy + \overline{x}z + xyz + \overline{x}yz \\
                &= (xy + xyz) + (\overline{x}z + \overline{x}yz) \\
                &= xy(1 + z) + \overline{x}z(1 + y) \\
                &= xy + \overline{x}z
                \end{aligned}$$
        \end{proof}


        \subsubsection*{Secondary Boolean Operators}
        \begin{definition}[Secondary Boolean Operators]
            Secondary operators are derived from basic operators:
            \begin{itemize}
                \item Material conditional: \( x \rightarrow y = \lnot x \lor y \)
                \item Material biconditional: \( x \leftrightarrow y = (x \land y) \lor (\lnot x \land \lnot y) \)
                \item Exclusive OR (XOR): \( x \oplus y = \lnot(x \leftrightarrow y) = (x \lor y) \land (\lnot x \lor \lnot y) = (x \land \lnot y) \lor (\lnot x \land y) \)
                \end{itemize}
        \end{definition}

        Their truth tables for the secondary operation are below:
        \begin{table}[H]
        \centering
        \begin{tabular}{cc|c|c|c}
        \hline
        \( x \) & \( y \) & \( x \rightarrow y \) & \( x \leftrightarrow y \) & \( x \oplus y \) \\
        \hline
        0 & 0 & 1 & 1 & 0 \\
        1 & 0 & 0 & 0 & 1 \\
        0 & 1 & 1 & 0 & 1 \\
        1 & 1 & 1 & 1 & 0 \\
        \hline
        \end{tabular}
        \caption{Truth values of material conditional, biconditional, and XOR for all possible inputs.}
        \end{table}
        The $\rightarrow$ operation holds that $x \rightarrow x = 1$. This expression means
        that when $x=y$, then $x\rightarrow y$ must be true. The complete form of secondary operators
        and the implication identity will be the last Boolean identity to be covered in this chapter.
        \begin{enumerate}
            \item Material Conditional (\(\rightarrow\)):
            The material conditional \( x \rightarrow y \) is read as "if \( x \) then \( y \)" or "\( x \) implies \( y \)". It represents the logical implication. The truth value of \( x \rightarrow y \) is false only when \( x \) is true and \( y \) is false; in all other cases, it is true. This is a bit counter-intuitive when \( x \) is false because the implication will be true regardless of the value of \( y \). It can be expressed using basic operations as \( \lnot x \lor y \).
            
            \item Material Biconditional (\(\leftrightarrow\)):
            The material biconditional \( x \leftrightarrow y \), also known as logical equivalence, is the operation that is true when \( x \) and \( y \) have the same truth values, and false otherwise. It's often read as " \( x \) if and only if \( y \)". It can be formulated as \( (x \land y) \lor (\lnot x \land \lnot y) \), meaning both \( x \) and \( y \) are true, or both are false.
            
            \item Exclusive OR (XOR):
            The exclusive OR \( x \oplus y \) is true when \( x \) and \( y \) have different truth values — that is, one is true and the other is false. It differs from the regular OR operation in that \( x \oplus y \) is false when both \( x \) and \( y \) are true. It can be represented as \( (x \land \lnot y) \lor (\lnot x \land y) \).
        \end{enumerate}
        
        With these basic identities and operators, we can prove more interesting Boolean theorems.
        Below are more theorems to be proved as example. Maybe you are ready to draw a truth table to prove
        them. However, with these basic Boolean Laws, we actually don't have to do that, but prove it
        by using Algebra techniques.
        \begin{example}
            Prove the following identities using the table given earlier.
        \begin{table}[ht]
            \centering
            \begin{tabular}{ll}
            \toprule
            \textbf{Expression} & \textbf{Name} \\
            \midrule
            \( (x \rightarrow \text{False}) = (\lnot x) \) & $\lnot$ as \(\rightarrow\) \\
            \( (x \rightarrow y) = (\lnot y \rightarrow \lnot x) \) & contrapositive \\
            \( ((x \rightarrow y) \land (x \rightarrow z)) = (x \rightarrow (y \land z)) \) & implication \\
            \( ((x \rightarrow y) \land (\lnot x \rightarrow y)) = y \) & absurdity \\
            \( (x \rightarrow (\lnot x)) = (\lnot x) \) & contradiction \\
            \bottomrule
            \end{tabular}
        \end{table}
        \end{example}

        \begin{proof}
            To prove $\lnot$ as \(\rightarrow\), we just need to use the definition of $
            \rightarrow$.
            $$x\rightarrow\text{ False}= \lnot x \lor False = \lnot x + 0 = \lnot x \ \text{(OR Identity Law)}$$
            
            The contrapositive identity states that \( (x \rightarrow y) \) is logically equivalent to \( (\lnot y \rightarrow \lnot x) \).
            \begin{align*}
            x \rightarrow y &\equiv \lnot x \lor y \\
                &\equiv y \lor \lnot x\ \text{(Commutative Law)}\\
            &\equiv \lnot y \rightarrow \lnot x
            \end{align*}
            \begin{remark}
                This proof shows why proof by contrapositive (mentioned in chapter1) is correct.
            \end{remark}
            The implication identity can be proved by showing that \( (x \rightarrow y) \land (x \rightarrow z) \) is equivalent to \( x \rightarrow (y \land z) \).
            \begin{align*}
            (x \rightarrow y) \land (x \rightarrow z) &\equiv (\lnot x \lor y) \land (\lnot x \lor z) \\
            &\equiv \lnot x \lor (y \land z) \\
            &\equiv x \rightarrow (y \land z)
            \end{align*}

            The absurdity identity states that \( (x \rightarrow y) \land (\lnot x \rightarrow y) \) is equivalent to \( y \).
            \begin{align*}
            (x \rightarrow y) \land (\lnot x \rightarrow y) &\equiv (\lnot x \lor y) \land (x \lor y) \\
            &\equiv y \lor (x \land \lnot x) \\
            &\equiv y \lor \text{False} \\
            &\equiv y
            \end{align*}

            The contradiction identity states that \( x \rightarrow (\lnot x) \) is equivalent to \( \lnot x \).
            \begin{align*}
            x \rightarrow (\lnot x) &\equiv \lnot x \lor (\lnot x) \\
            &\equiv \lnot x
            \end{align*}
        \end{proof}
        \begin{remark}
            Some may feel confused by the notation of Boolean expression we use in this chapter.
            Sometimes we use logic notation, while sometimes use algebra operation. Both are ok,
            it's really up to your preference.
        \end{remark}
        When you look through the whole process, you'll find that we are actually doing the same
        thing as simplifying Boolean expressions, where all the laws could be used so that we don't
        need to draw truth table again and again.

        \subsection{Boolean Function}
        Now we have learned most of the rules of Boolean operations; we use the rest of this section
        to bring you some further ideas of Boolean operation. Just like we can take a normal algebra
        expression, such as $2x+3$, as a function $f(x)=2x+3$; we can define \textbf{Boolean Function}
        with Boolean operation.
        \begin{definition}[Boolean Function]
        A Boolean function is a function that returns a Boolean value (true or false) for each possible combination of Boolean inputs. 
        $f: B^n \rightarrow B$ is a function from the set of  $n$-tuples of Boolean values to the set of Boolean values, where  $B = \{0, 1\}$.
        \end{definition} 




        \subsection{Exercises}
        \begin{exercise}
            Use table \ref{algelaw} and $1+1=2$ to show that $(x+x)=(2\times x)$.
        \end{exercise}
        \begin{proof}
            \begin{align*}
                ( 2\times x) & =( 1+1) \times x\ ( by\ 1+1=2)\\
                & =\ 1\times x+1\times x\ ( Distributive\ Law)\\
                & =x+x\ ( Multiplication\ Identity)
            \end{align*}
        \end{proof}
        \begin{exercise}
            Show that $((-1)\times x)+x=0$ using the table.
        \end{exercise}
        \begin{proof}
            \begin{align*}
                (( -1) \times x) +x & =x( -1+1) \ ( Distributive\ Law)\\
                 & =\ x\times 0\ ( Addition\ Inverse)\\
                 & =0\ ( Multiplication\ Property\ of\ Zero)
            \end{align*}
        \end{proof}
        \begin{exercise}
            Show that $(x+(((-1)\times (x+y))+z)) + y =z$, you may use the conclusion of previous
            exercise.
        \end{exercise}
        \begin{proof} 
            To make it clear, we use square and curly bracket for different layers of parenthesis.
            \begin{align*}
                \  & \{x+[((-1) \times ( x+y)) +z]\} +y\\
                & =\{x+[(( -1) \times x) +(( -1) \times y)] +z\} +y\ ( Distributive\ Law\ )\\
                & =\{x+(( -1) \times x) +[(( -1) \times y) +z]\} +y\ ( Associative\ Law\ of\ Addition)\\
                & =\{(( -1) \times x) +x+\ [(( -1) \times y) +z]\} +y( Commutative\ Law\ of\ Addition)\\
                & =0+\{[(( -1) \times y) +z] +y\}\ ( Previous\ Conclusion)\\
                & =\{[(( -1) \times y) +z] +y\} +0\ ( Commutative\ Law\ of\ Addition)\\
                & =\{[(( -1) \times y) +z] +y\} \ ( Addition\ Identity)\\
                & =(( -1) \times y) +( z+y) \ ( Associative\ Law\ of\ Addition)\\
                & =( z+y) +(( -1) \times y) \ ( Commutative\ Law\ of\ Addition)\\
                & =z+[ y+(( -1) \times y)] \ ( Associative\ Law\ of\ Addition)\\
                & =z+[(( -1) \times y) +y] \ ( Commutative\ Law\ of\ Addition)\\
                & =z+0\ ( Previous\ Conclusion)\\
                & =z\ ( Addition\ Identity)
            \end{align*}
            
        \end{proof}
        
        \begin{exercise}
            Use truth table to show that De Morgan's Law is correct.
        \end{exercise}
        \begin{proof}
            De Morgan's First Law: \( \lnot (A \land B) = \lnot A \lor \lnot B \)
            
            \begin{tabular}{cc|c|c|c}
            \toprule
            \( A \) & \( B \) & \( A \land B \) & \( \lnot (A \land B) \) & \( \lnot A \lor \lnot B \) \\
            \midrule
            0 & 0 & 0 & 1 & 1 \\
            0 & 1 & 0 & 1 & 1 \\
            1 & 0 & 0 & 1 & 1 \\
            1 & 1 & 1 & 0 & 0 \\
            \bottomrule
            \end{tabular}

            \vspace{2em} 

            De Morgan's Second Law: \( \lnot (A \lor B) = \lnot A \land \lnot B \)

            \begin{tabular}{cc|c|c|c}
            \toprule
            \( A \) & \( B \) & \( A \lor B \) & \( \lnot (A \lor B) \) & \( \lnot A \land \lnot B \) \\
            \midrule
            0 & 0 & 0 & 1 & 1 \\
            0 & 1 & 1 & 0 & 0 \\
            1 & 0 & 1 & 0 & 0 \\
            1 & 1 & 1 & 0 & 0 \\
            \bottomrule
            \end{tabular}
        \end{proof}

        \begin{exercise}
            Use truth table to show that Absorption Law is correct.
        \end{exercise}
        \begin{proof}
            \begin{tabular}{cc|c|c}
                \hline
                \( x \) & \( y \) & $x \lor y$&\( x \land (x \lor y) \) \\
                \hline
                T & T & T & T \\
                T & F & T & T \\
                F & T & F & F \\
                F & F & F & F \\
                \hline
            \end{tabular}
            \quad
            \begin{tabular}{cc|c|c}
                \hline
                \( x \) & \( y \)&$x \land y$ & \( x \lor (x \land y) \) \\
                \hline
                T & T& T& T \\
                T & F& F& T \\
                F & T& F& F \\
                F & F& F& F \\
                \hline
            \end{tabular}
        \end{proof}

        \begin{exercise}
            Show the truth table of $\displaystyle x\lor (( \lnot y) \land ( \lnot z))$.
        \end{exercise}
        \textbf{Hint:} For Boolean expression of 3 variable, there will be more combinations.
        
        \textbf{Solution:}

        \begin{table}[ht]
            \centering
            \caption{Truth table for the expression \( x \lor ((\lnot y) \land (\lnot z)) \)}
            \begin{tabular}{ccc|c|c|c|c}
            \hline
            \( x \) & \( y \) & \( z \) & \( \lnot y \) & \( \lnot z \)& $(\lnot y) \land (\lnot z)$ & \( x \lor ((\lnot y) \land (\lnot z)) \) \\
            \hline
            0 & 0 & 0 & 1 & 1 &1& 1 \\
            0 & 0 & 1 & 1 & 0 &0& 0 \\
            0 & 1 & 0 & 0 & 1 &0& 0 \\
            0 & 1 & 1 & 0 & 0 &0& 0 \\
            1 & 0 & 0 & 1 & 1 &0& 1 \\
            1 & 0 & 1 & 1 & 0 &0& 1 \\
            1 & 1 & 0 & 0 & 1 &0& 1 \\
            1 & 1 & 1 & 0 & 0 &0& 1 \\
            \hline
            \end{tabular}
 
        \end{table}

        \begin{exercise}
            Given the expressions:
            \begin{align*}
            1. & \quad (x \rightarrow y) \land (\lnot x \rightarrow \lnot y) \\
            2. & \quad ((\lnot x) \lor y) \land (x \lor (\lnot y)) \\
            3. & \quad \lnot((x \land (\lnot y)) \lor ((\lnot x) \land y)) \\
            4. & \quad \lnot((x \lor y) \land (\lnot x \lor \lnot y)) \\
            \end{align*}
            Show that they are equivalent and find the common (simplified) form of the four expression
        \end{exercise}
        \begin{proof}
            \begin{align*}
                (x \rightarrow y) \land (\neg x \rightarrow \neg y) & \equiv (\neg x \lor y) \land (x \lor \neg y) && \text{(Implication equivalence)} \\
                & \equiv (x \lor \neg y) \land (\neg x \lor y) && \text{(Commutativity of OR)} \\
                & \equiv (x \lor \neg y) \land (y \lor \neg x) && \text{(Commutativity of AND)}
                \end{align*}

                \begin{align*}
                    ((\neg x) \lor y) \land (x \lor (\neg y)) & \equiv (y \lor \neg x) \land (x \lor \neg y) && \text{(Commutativity of OR)} \\
                    & \equiv (x \lor \neg y) \land (y \lor \neg x) && \text{(Commutativity of AND)}
                    \end{align*}
            
                    \begin{align*}
                        \neg((x \land (\neg y)) \lor ((\neg x) \land y)) & \equiv \neg(x \land (\neg y)) \land \neg((\neg x) \land y) && \text{(De Morgan's laws)} \\
                        & \equiv (\neg x \lor y) \land (x \lor \neg y) && \text{(De Morgan's laws)} \\
                        & \equiv (x \lor \neg y) \land (y \lor \neg x) && \text{(Commutativity of OR and AND)}
                        \end{align*}

                        \begin{align*}
                            \neg((x \lor y) \land (\neg x \lor \neg y)) & \equiv \neg(x \lor y) \lor \neg(\neg x \lor \neg y) && \text{(De Morgan's laws)} \\
                            & \equiv (\neg x \land \neg y) \lor (x \land y) && \text{(De Morgan's laws)} \\
                            & \equiv (x \land y) \lor (\neg x \land \neg y) && \text{(Commutativity of OR)} \\
                            & \equiv (x \lor \neg y) \land (y \lor \neg x) && \text{(Distribution)}
                            \end{align*}
                            

            These expressions could be simplified to $(\lnot y \lor x)\land(\lnot x\lor y)$, or
            $(y\rightarrow x)\land(x\rightarrow y)$, which can also be understood as $x$ and $y$
            cannot be true of false in the same time.
        \end{proof}

        \begin{exercise}
            Show that the concensus theorem's two forms are equivalent.
                $$xy\lor\bar{x}z\lor yz=xy\lor\bar{x}z$$
            and
                $$(x\lor y)(\bar{x}\lor z)(y\lor z)=(x\lor y)(\bar{x}\lor z)$$
            are equivalent.
        \end{exercise}
        \textbf{Solution:}
        We can simplify the product form to the standard form easily.
        \begin{align*}
            (x \lor y)(\overline{x} \lor z)(y \lor z) &= x\overline{x} \lor xz \lor y\overline{x} \lor yz \lor xy \lor xz \lor y^2 \lor yz \\
            &= 0 \lor xz \lor y\overline{x} \lor yz \lor xy \lor xz \lor y \lor yz \\
            &= xz \lor y\overline{x} \lor yz \lor xy \\
            &= xy \lor xz \lor y\overline{x}
        \end{align*}

    \section{Predicates and Quantifiers}

    \section{Digital Circuits of Logic}

    \section{Logic of Deduction and Induction}