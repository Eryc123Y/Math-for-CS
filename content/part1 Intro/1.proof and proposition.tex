%----------------------------------------------------------------------------------------

\chapterimage{chap1.png} % Chapter heading image
\chapterspaceabove{4.25cm} % Whitespace from the top of the page to the chapter title on chapter pages
\chapterspacebelow{7.25cm} % Amount of vertical whitespace from the top margin to the start of the text on chapter pages

%------------------------------------------------

\chapter{Mathematical Proof Strategy}
The first Chapter of this book focus only on the most essential part of mathematics, proofs. Proofs are the very essence of mathematics, 
serving as the definitive tool for establishing the truth within this discipline of absolute certainty. Unlike empirical sciences, where conclusions are 
drawn based on observation and experimentation subject to uncertainties, mathematical proofs provide incontrovertible evidence that a statement is true. 
They are the architects of mathematical theory, constructing a framework of knowledge that is both logical and immutable. Through proofs, we not 
only validate conjectures but also weave a tapestry of interconnected truths, each supported by the unshakable foundation of previously proven results. 
This interconnectedness ensures that mathematical knowledge, once proven, becomes a permanent addition to the collective human understanding, transcending time 
and offering a universal language spoken by all cultures in the language of logic and reason.

%------------------------------------------------
\section{Propositions}
The angles in a triangle add up to 180 degrees; the sum of any two even numbers is even. Statement as such is so common in mathematics, which we call proposition.
	\begin{definition}[Proposition]
		A proposition is a declarative sentence that is either true or false. 
		Propositions are the fundamental building blocks of mathematical reasoning, as they can be clearly judged to be true or false.
	\end{definition}
	Propositions have the following characteristics:
	\begin{enumerate}
		\item \textbf{Definiteness:} Propositions must be clear and unambiguous so that they can be definitively judged to be true or false.
		\item \textbf{Exclusivity:} Propositions admit no middle ground between true and false.
		\item \textbf{Objectivity:} Propositions represent objective facts, not subjective opinions or questions.
	\end{enumerate}

\noindent	Propositions are essential and crucial for proof, this is because
	\begin{itemize}
		\item \textbf{Foundation:} Propositions are the foundation upon which logical reasoning is built.
		\item \textbf{Validity:} Proving the validity of a proposition reinforces the truthfulness of a statement.
		\item \textbf{Interconnectedness:} Propositions are interconnected; proving one can help establish the truth of others.
	\end{itemize}

	In mathematical discourse, the clarity and truth of propositions are paramount. Understanding the nature of propositions (truth, falsity, and reasoning) is essential for constructing and understanding mathematical arguments.	
Commonly, Propositions are categorized by the \textbf{truth value}, which we will discuss
in Boolean Algebra, and we call a proposition true proposition when the statement is factually correct, while false
proposition, vice versa. Most importantly: \textbf{If a proposition is true, it can be proven}.
\section*{Converse, Inverse, and Contrapositive}
In the study of logic, particularly within the context of mathematical reasoning, we come across several important concepts that relate to conditional statements. A conditional statement is typically of the form "If \(P\), then \(Q\)", denoted \(P \rightarrow Q\). Here we define and discuss the converse, inverse, and contrapositive of a conditional statement.

\subsection*{Converse}
The converse of a statement flips the hypothesis and the conclusion. For the statement \(P \rightarrow Q\), the converse is \(Q \rightarrow P\). It is important to note that the truth of a converse is not necessarily the same as the truth of the original statement.

\subsection*{Inverse}
The inverse of a statement negates both the hypothesis and the conclusion. For the statement \(P \rightarrow Q\), the inverse is \(\neg P \rightarrow \neg Q\), where \(\neg\) denotes negation. Similar to the converse, the truth of the inverse is not dependent on the truth of the original statement.

\subsection*{Contrapositive}
The contrapositive of a statement negates and flips the hypothesis and the conclusion. For the statement \(P \rightarrow Q\), the contrapositive is \(\neg Q \rightarrow \neg P\). Unlike the converse and the inverse, the truth of the contrapositive is always the same as the truth of the original statement. This property is often used in mathematical proofs, particularly in \textbf{proofs by contradiction}, which we will discuss in successive sections.

\section{Direct Proof}
	\subsection*{Framework of Direct Proof}
	A direct proof is a fundamental method in mathematics used to establish the truth of a
	given statement, typically a theorem or proposition. It is characterized by a straightforward
	and logical progression from known facts or axioms to the conclusion. The approach
	of a direct proof is to assume that the premises (initial assumptions or known truths)
	are correct and then to use logical reasoning and established mathematical principles to
	demonstrate that the conclusion necessarily follows from these premises. Here's a general
	structure of how a direct proof works:
	\begin{enumerate}
		\item Start with Known Facts or Assumptions: Begin with what is already known or
		assumed to be true. These can be definitions, previously proven theorems, or given
		premises.
		\item Logical Argumentation: Use logical reasoning and mathematical operations to derive
		new information from these known facts. This process often involves applying
		definitions, using the properties of mathematical operations, and invoking previously
		established theorems.
		\item Arrive at the Conclusion: The final step is to show that the statement you set out
		to prove is a logical consequence of the initial assumptions. The conclusion should
		follow naturally and unavoidably from the previous steps.
	\end{enumerate}
	To illustrate:
    \begin{example}
	
		Prove that if \( n \) is an odd integer, then \( n^2 \) is also odd.
	
    \end{example}

    \begin{proof}
		Assume \( n \) is an odd integer. By definition, an odd integer can be written as \( n = 2k + 1 \) where \( k \) is an integer.
		
		Squaring both sides of this equation, we get
		\[ n^2 = (2k + 1)^2 = 4k^2 + 4k + 1 = 2(2k^2 + 2k) + 1. \]
		
		Let \( m = 2k^2 + 2k \), which is an integer since it is a sum of integers. Therefore, we can express \( n^2 \) as
		\[ n^2 = 2m + 1. \]
		
		This is the form of an odd integer. Thus, we have shown that if \( n \) is an odd integer, then \( n^2 \) is also odd.
		\end{proof}
  
		We can see that the idea of direct proof is something everyone can understand, however,
		what is really challenging is to find the specific know facts to assist the proof. Here
		are more exercises on direct proof for reference.
        
		\subsection{Exercises}
		\begin{exercise}
			Prove that if \( n \) is an even integer, then \( n^2 \) is also even.
		\end{exercise}
		Hint: follow the procedure of the example of odd number, using the definition of even integer.
		\begin{proof}
        Assume \( n \) is an even integer. By definition, an even number can be expressed as \( n = 2k \) where \( k \) is an integer.
        
        Squaring both sides of this equation, we get:
        \[ n^2 = (2k)^2 = 4k^2 = 2(2k^2). \]
        
        Since \( 2k^2 \) is an integer (let's call it \( m \)), we can express \( n^2 \) as \( 2m \), which is the definition of an even number.
        
        Hence, if \( n \) is an even integer, then \( n^2 \) is also even.
        \end{proof}
        \begin{exercise}
			Prove that the sum of two even integers is even.
		\end{exercise}
		Hint: Use the definition of an even integer.
        \begin{proof}
        Let \( a \) and \( b \) be even integers. By the definition of even integers, there exist integers \( k \) and \( m \) such that \( a = 2k \) and \( b = 2m \).
        
        The sum of \( a \) and \( b \) is:
        \[ a + b = 2k + 2m = 2(k + m). \]
        
        Let \( n = k + m \), which is an integer since both \( k \) and \( m \) are integers. Hence, \( a + b = 2n \).
        
        Since \( a + b \) is two times an integer, it is even by definition. This concludes the proof.
        \end{proof}
		
		\begin{exercise}\label{ex1.3}
			Prove that for any positive integer \( n \), \( n^3 + 2n \) is divisible by 3.
		\end{exercise}
			Hint: Factorize \( n^3 + 2n \) and use the properties of divisibility.
        \begin{proof}
         Consider the expression \( n^3 + 2n \). This can be factored as:
        \[ n^3 + 2n = n(n^2 + 2) = n(n^2 - 1 + 3) = n(n+1)(n-1)+3n. \]
        
        Notice that \( n^2 + 2 \) can be written as \( (n^2 - 1) + 3 = (n+1)(n-1) + 3 \). The terms \( n + 1 \), $n$, and \( n - 1 \) are three consecutive integers, so one of them must be multiple of 3. Therefore, $n(n+1)(n-1)$ can be denoted by $3m$, where $m$ is a positive integer, and the expression is equivalent to $3(m+n)$. 
        
        Hence, for any positive integer \( n \), \( n^3 + 2n \) is divisible by 3.
        
        \end{proof}
    \begin{remark}
        Considering there are still concepts that we haven't covered in Boolean algebra and number theory, the exercises in this chapter will be more basic than other chapters. What's really important in this chapter is not to finish difficult proof, but grasp the idea of proving.
    \end{remark}

\section{Proof by Cases}
    Sometimes, to draw a specific conclusion, we must make multiple or even infinite assumptions (The latter we will discuss in \textbf{Strong Mathematical induction}). This proof pattern stands alone from direct proof, as when we use this method, the proof itself could still be direct or indirect.
    \subsection*{Framework of Proof by Cases}
        \begin{proof}
        The proof is by cases. We consider each possible case and show that the theorem holds in each case.
        
        \textbf{Case 1:} [Description of Case 1]
        \begin{proof}[Proof of Case 1]
        Present the proof for Case 1 here. Use logical reasoning and mathematical principles to demonstrate that the theorem holds under the assumptions of Case 1.
        \end{proof}
        
        \textbf{Case 2:} [Description of Case 2]
        \begin{proof}[Proof of Case 2]
        Similarly, present the proof for Case 2 here, showing that the theorem is valid in this scenario as well.
        \end{proof}

% Add more cases as necessary

\textit{Final Conclusion:} Since the theorem holds in all possible cases, we conclude that the theorem is proved.
\end{proof}

    Here is an example of proof by cases.
    \begin{example}
        
            Prove that the sum of any three consecutive integers is divisible by 3.
            
    \end{example}

    \begin{proof}
            Let the three consecutive integers be \( n \), \( n+1 \), and \( n+2 \), where \( n \) is an integer. We consider three cases for \( n \), based on the division by 3.
            
            \textbf{Case 1:} \( n \) is of the form \( 3k \) for some integer \( k \).
            
            \textbf{Case 2:} \( n \) is of the form \( 3k + 1 \) for some integer \( k \).
            
            \textbf{Case 3:} \( n \) is of the form \( 3k + 2 \) for some integer \( k \).
            
            In each case, the sum of \( n \), \( n+1 \), and \( n+2 \) can be expressed as:
            
            For Case 1: \( (3k) + (3k + 1) + (3k + 2) = 9k + 3 = 3(3k + 1) \), it is divisible by 3.
            
            For Case 2: \( (3k + 1) + (3k + 2) + (3k + 3) = 9k + 6 = 3(3k + 2) \), it is divisible by 3.
            
            For Case 3: \( (3k + 2) + (3k + 3) + (3k + 4) = 9k + 9 = 3(3k + 3) \), it is divisible by 3.
            
            In each of the three cases, the sum is a multiple of 3. Therefore, we conclude that the sum of any three consecutive integers is divisible by 3.
            \end{proof}
            
\begin{remark}
    As mentioned in direct proof, the idea of proving is always simple and easy to understand. For proof by cases, the toughest part is also finding all the cases needed to prove a certain conclusion.
\end{remark}
\subsection{Exercises}
\begin{exercise}
			Prove that the product of two consecutive integers is always even.
\end{exercise}
		Hint: Express the consecutive integers as \( n \) and \( n + 1 \).
\begin{proof}
Consider two consecutive integers, \( n \) and \( n + 1 \), where \( n \) is an integer. The product of these two integers is \( n(n + 1) \).

We have two cases to consider:

\textbf{Case 1:} If \( n \) is even, then it can be written as \( n = 2k \) for some integer \( k \). The product is then \( n(n + 1) = 2k(2k + 1) \), which is even because it is divisible by 2.

\textbf{Case 2:} If \( n \) is odd, then \( n + 1 \) is even. In this case, \( n + 1 = 2k \) for some integer \( k \). The product is then \( n(n + 1) = n(2k) \), which is even because it is divisible by 2.

In either case, the product of \( n \) and \( n + 1 \) is even. Therefore, the product of two consecutive integers is always even.
\end{proof}
\begin{exercise}
	Prove that if \( n \) is an integer, then \( n(n+1) \) is even.
\end{exercise}
Hint: Consider the cases where n is even and where n is odd separately.
\begin{proof}
We will consider two cases based on the parity of \( n \).

\textbf{Case 1:} \( n \) is even.

Let \( n = 2k \) for some integer \( k \). Then \( n(n+1) = 2k(2k+1) \). Since \( 2k \) is even, the product \( 2k(2k+1) \) is also even because the product of an even number and any other integer is even.

\textbf{Case 2:} \( n \) is odd.

Let \( n = 2k+1 \) for some integer \( k \). Then \( n(n+1) = (2k+1)(2k+2) \). Here \( (2k+2) \) is even, and thus the product \( (2k+1)(2k+2) \) is even because the product of an even number and any other integer is even.

In both cases, whether \( n \) is even or odd, \( n(n+1) \) is even. Hence, we have shown that for any integer \( n \), the product \( n(n+1) \) is always even.
\end{proof}

\begin{exercise}
Prove that for any integer \( n \), \( n^2 + 4 \) cannot be a prime number if \( n > 1 \).
\end{exercise}
Hint: Consider the cases where n is even and where n is odd.
\begin{proof}
Consider two cases based on the parity of \( n \).

\textbf{Case 1:} \( n \) is even.

Let \( n = 2k \) for some integer \( k \). Then \( n^2 + 4 \) becomes \( (2k)^2 + 4 = 4k^2 + 4 = 4(k^2 + 1) \). Since \( k^2 + 1 \) is an integer greater than 1, \( 4(k^2 + 1) \) is not prime because it has factors other than 1 and itself, namely 4 and \( k^2 + 1 \).

\textbf{Case 2:} \( n \) is odd.

Let \( n = 2k + 1 \) for some integer \( k \). Then \( n^2 + 4 \) becomes \( (2k + 1)^2 + 4 = 4k^2 + 4k + 1 + 4 = 4k^2 + 4k + 5 \). This can be factored as \( (2k + 1)(2k + 3) \), which are two consecutive odd numbers. Since both \( 2k + 1 \) and \( 2k + 3 \) are factors greater than 1 but not the expression itself, the product is not prime.
Therefore, in both cases, whether \( n \) is even or odd, \( n^2 + 4 \) is not a prime number.
\end{proof}

\begin{exercise}

Prove that for any integer \( n \), the expression \( n^5 - n \) is divisible by 30.

\end{exercise}
Hint: Consider divisibility of 5 and 6 (2 and 3), and prove by cases accordingly. 
\begin{proof}
To prove that \( n^5 - n \) is divisible by 6, it suffices to show that it is divisible by 2, 3, and 5.

First, observe that \( n^5 - n = n(n^4 - 1) \).

\textbf{Divisibility by 2:} 
\begin{itemize}
  \item If \( n \) is even, then \( n^5 - n \) is clearly even since it is a multiple of \( n \).
  \item If \( n \) is odd, \( n^4 - 1 \) is even because \( n^4 \) is odd and \( n^4 - 1 \) is one less than an odd number, making it even.
\end{itemize}

\textbf{Divisibility by 3:} \\ 
Any integer \( n \) is either a multiple of 3, one more than a multiple of 3, or two more than a multiple of 3. That is, \( n \) can be written as \( 3k \), \( 3k + 1 \), or \( 3k + 2 \) for some integer \( k \).

\begin{itemize}
    \item If \( n = 3k \), then \( n^5 - n = (3k)^5 - 3k \) is clearly a multiple of 3.
    \item If \( n = 3k + 1 \), then \( n^5 - n = (3k + 1)^5 - (3k + 1) \). Expanding \( (3k + 1)^5 \) using the binomial theorem, all terms except the last are multiples of 3, and the last term \( 1^5 \) minus \( 3k + 1 \) leaves \( -3k \), which is a multiple of 3.
    \item If \( n = 3k + 2 \), a similar expansion shows that \( n^5 - n \) is a multiple of 3.
\end{itemize}

Thus, \( n^5 - n \) is divisible by 3.

\textbf{Divisibility by 5:} 
For any integer \( n \), we can express \( n \) in one of the following forms, where \( k \) is an integer:
\begin{enumerate}
    \item \( n = 5k \) (where \( n \) is a multiple of 5)
    \item \( n = 5k + 1 \)
    \item \( n = 5k + 2 \)
    \item \( n = 5k + 3 \)
    \item \( n = 5k + 4 \)
\end{enumerate}

We will prove that \( n^5 - n \) is divisible by 5 for each case. Since we are working modulo 5, we only need to consider the last digit of \( n \) when raised to the fifth power due to the cyclicity of powers modulo 5.

\begin{itemize}
    \item For \( n = 5k \): \( n^5 - n = (5k)^5 - 5k \). Clearly, both terms are divisible by 5.
    \item For \( n = 5k + 1 \): \( n^5 - n = (5k + 1)^5 - (5k + 1) \). When expanded, each term of \( (5k + 1)^5 \) except the last will contain a factor of \( 5k \) and thus be divisible by 5. The last term \( 1^5 = 1 \), and subtracting \( 5k + 1 \) leaves a result which is still divisible by 5.
    \item For \( n = 5k + 2 \): \( n^5 - n = (5k + 2)^5 - (5k + 2) \). Again, each term in the expansion of \( (5k + 2)^5 \), except the last, will be divisible by 5. The last term will be \( 2^5 = 32 \), which is congruent to 2 modulo 5, and subtracting \( (5k + 2) \) leaves a result divisible by 5.
    \item For \( n = 5k + 3 \) and \( n = 5k + 4 \), a similar argument holds as for \( n = 5k + 1 \) and \( n = 5k + 2 \), respectively.
\end{itemize}

In all cases, \( n^5 - n \) is divisible by 5. This completes the proof for Case 3.

Therefore, \( n^5 - n \) is divisible by 2, 3, and 5, and hence by 30.
\end{proof}


%----------------------------------------------------------------------

\section{Indirect Proof}
    An indirect proof is a powerful method in mathematics used to prove a statement by showing that the negation leads to a contradiction or by proving the contrapositive of the statement. We will discuss proof by contradiction and proof by contrapositive separately.

\subsection{Proof by Contradiction}

Proof by contradiction is a critical reasoning technique used extensively in mathematics and logic. Its importance lies in its ability to confirm the truth of a statement by demonstrating that assuming the opposite leads to an illogical or impossible conclusion. This method is particularly valuable because it can sometimes prove assertions that are otherwise difficult to demonstrate directly. It is a cornerstone of mathematical reasoning and is often used to establish fundamental theorems that form the bedrock of mathematical theories. The use of this technique highlights the rigorous nature of mathematical proof and emphasizes the importance of logical consistency within mathematical frameworks.

Here is an abstract template for proof by contradiction:

Suppose, for the sake of contradiction, that the statement we wish to prove is false.

\begin{enumerate}
    \item State the negation of the theorem or proposition you are trying to prove.
    \item Use logical reasoning and established mathematical principles to derive consequences of this assumption.
    \item Show that these consequences lead to a contradiction, something that is known to be false or violates a basic principle of mathematics.
    \item Conclude that since the assumption leads to a contradiction, the negation must be false, and therefore, the original statement is true.
\end{enumerate}

Contradiction could be used to proof many interesting conclusions in mathematics, here is a classical example that prove the length of a diagonal in a square with side length 1 is irrational number.
\begin{example}
    Prove that the length of a diagonal in a square with side length 1 is irrational number
\end{example}
\begin{proof}
    Assuming $\sqrt{2}$ is rational, we can express $x$ as $\frac{p}{q}$ where $p$ and $q$ are integers with no common factors, and we have:
\begin{remark}
        Rational numbers must be able to be expressed in the form of $\frac{p}{q}$, where $p$ and $q$ are both natural numbers with no common factors other than 1.
    \end{remark}
    Therefore, we have $\frac{p^2}{q^2} = 2$. Thus,  $p^2 = 2q^2$. This means that 
\begin{equation}
\left(\frac{p}{q}\right)^2 = 2.
\end{equation}

This leads to:

\begin{equation}
p^2 = 2q^2.
\end{equation}

This implies that $p^2$ is an even number since it is twice some integer. Therefore, $p$ must also be even, as the square of an odd number cannot be even. Let $p = 2r$, then equation (1.2) becomes:

\begin{equation}
(2r)^2 = 2q^2,
\end{equation}
which simplifies to:

\begin{equation}
2r^2 = q^2.
\end{equation}

This indicates that $q^2$ is also even, and hence $q$ must be even. However, this is a contradiction because if both $p$ and $q$ are even, they are not coprime, which violates the initial assumption that $p$ and $q$ have no common factors. Therefore, our initial assumption that $\sqrt{2}$ is rational must be false. Hence, $\sqrt{2}$ is irrational.

Thus, we conclude that the number $\sqrt{2}$ is irrational.
    
\end{proof}
With this example, we can see that this is a very rigorous proving method that could be used to prove propositions that we have to prove indirectly. Do take note that using this method requires you to make clear assumption and find contradiction accurately. 
\subsection*{Exercises}
\begin{exercise}
    Prove that $\sqrt{5}$ is irrational.
\end{exercise}
Hint: Refer to the example. Consider the following lemma:
\noindent If an integer $p$ can be expressed as the product of two integers $a$ and $b$, that is, $p = ab$, then $p$ is not a prime number.
\begin{proof}
    We will prove by contradiction that the square root of 5 is an irrational number. Suppose $\sqrt{5}$ is rational, which means it can be expressed as a fraction $\frac{p}{q}$, where $p$ and $q$ are coprime integers, and $q \neq 0$. 

\begin{enumerate}
    \item We express $\sqrt{5}$ as a fraction in the lowest terms, so we have:
    \[
    \sqrt{5} = \frac{p}{q}
    \]
    where $p$ and $q$ have no common factors other than 1.

    \item Squaring both sides of the equation, we get:
    \[
    5 = \frac{p^2}{q^2}
    \]
    which implies:
    \[
    p^2 = 5q^2
    \]
    
    \item Since $p^2 = 5q^2$, $p^2$ is a multiple of 5, and hence $p$ must also be a multiple of 5, because the square of a number is only a multiple of 5 if the number itself is a multiple of 5. Let $p = 5k$, where $k$ is an integer.
    
    \item Substituting $p = 5k$ into $p^2 = 5q^2$, we get:
    \[
    (5k)^2 = 5q^2
    \]
    which simplifies to:
    \[
    25k^2 = 5q^2
    \]
    Dividing both sides by 5, we find:
    \[
    5k^2 = q^2
    \]
    
    \item This implies that $q^2$, and hence $q$, is also a multiple of 5.
    
    \item Therefore, both $p$ and $q$ are multiples of 5, which contradicts our initial assumption that they have no common factors other than 1. Hence, our initial assumption that $\sqrt{5}$ is rational must be false.
\end{enumerate}

Thus, we conclude that $\sqrt{5}$ cannot be expressed as a fraction, and therefore it is irrational.
\end{proof}

\begin{exercise}
Prove that there is no smallest positive rational number.
\end{exercise}
Hint: Assume that there is a smallest positive rational number and then show that you can find a smaller one, which leads to a contradiction.
\begin{proof}
Assume, for the sake of contradiction, that there exists the smallest positive rational number \( \frac{p}{q} \), where \( p \) and \( q \) are positive integers with no common factors other than 1, and \( q \neq 0 \).

Consider the rational number \( \frac{p}{2q} \). This number is positive since \( p \) and \( q \) are positive. It is also rational because it is the ratio of two integers. Moreover, \( \frac{p}{2q} \) is smaller than \( \frac{p}{q} \).

Since \( q \) is a positive integer and greater than 1 (because there are no smaller positive integers than 1), the inequality holds true. This means we have found a positive rational number smaller than our assumed smallest positive rational number, which is a contradiction.

Therefore, there cannot exist a smallest positive rational number, and our initial assumption is false.
\end{proof}

\begin{exercise}
Prove that if \( n \) is an integer and \( n^2 \) is even, then \( n \) is even.
\end{exercise}

\begin{proof}
Assume for the sake of contradiction that \( n \) is odd. According to the definition of an odd number, \( n \) can be expressed as \( n = 2k + 1 \) for some integer \( k \). We square this expression to get:
\[ n^2 = (2k + 1)^2 = 4k^2 + 4k + 1. \]
This expression can be rewritten as:
\[ n^2 = 2(2k^2 + 2k) + 1. \]
Since \( 2k^2 + 2k \) is an integer, the expression \( 2(2k^2 + 2k) \) is even, and adding 1 to an even number results in an odd number. Thus, \( n^2 \) is odd.

However, this contradicts our initial assumption that \( n^2 \) is even. Therefore, our assumption that \( n \) is odd must be false, and it follows that \( n \) must be even.
\end{proof}

\begin{exercise}
Prove that there are infinitely many prime numbers.
\end{exercise}

hint: Assume that there are only finitely many primes. Consider the product of all these primes plus one and analyze its prime factors to reach a contradiction.


\begin{proof}
Suppose for the sake of contradiction that there are only finitely many prime numbers. Let us list them as \( p_1, p_2, \ldots, p_n \). Consider the number \( P \) defined by the product of all these primes plus one:
\[ P = p_1p_2 \cdots p_n + 1. \]
By construction, \( P \) is greater than any of the listed prime numbers. 

Now, \( P \) must be divisible by some prime number (as every integer greater than 1 has a prime divisor). If \( P \) is divisible by any of the primes \( p_1, p_2, \ldots, p_n \), then there would be a remainder of 1, which is a contradiction because a prime number dividing \( P \) would leave no remainder. Therefore, \( P \) cannot be divided by any of the \( p_i \) without leaving a remainder. 

This implies that \( P \) must have a prime divisor that is not in our list, contradicting the assumption that we have listed all prime numbers. Thus, there must be more prime numbers than those in the list, and since our list was arbitrary, this means there are infinitely many prime numbers.
\end{proof}
\begin{exercise}
Prove that the sum of an irrational number and a rational number is irrational.
\end{exercise}

\begin{proof}
Assume for the sake of contradiction that the sum of an irrational number \( r \) and a rational number \( s \) is rational, and denote this sum as \( t \). Express \( t \) and \( s \) as the quotient of integers, such that \( t = \frac{a}{b} \) and \( s = \frac{c}{d} \), where \( a, b, c, d \) are integers and \( b, d \neq 0 \).

Since \( t \) is the sum of \( r \) and \( s \), we can write \( t = r + s \). Substituting the expressions for \( t \) and \( s \) gives us \( \frac{a}{b} = r + \frac{c}{d} \). Rearranging this equation to isolate \( r \) gives us \( r = \frac{a}{b} - \frac{c}{d} \). Combining the fractions, we find that \( r = \frac{ad - bc}{bd} \).

This expression shows that \( r \) is the quotient of two integers, which means \( r \) is rational. However, this contradicts our original statement that \( r \) is irrational.

Hence, we conclude by contradiction that the sum of an irrational number and a rational number cannot be rational, and therefore it is irrational.
\end{proof}



\subsection{Proof by Contrapositive} \label{contrapos}
Proof by contrapositive is a valid form of mathematical proof that is often used when direct proof or proof by contradiction is not as clear or straightforward. It's based on the logical equivalence between an implication and its contrapositive.

To prove a statement of the form “If \(P\), then \(Q\)” by contrapositive, we prove the equivalent statement “If not \(Q\), then not \(P\)”. It works as follows:

\begin{enumerate}
    \item Assume \(Q\) is false.
    \item Use logical reasoning to show that under this assumption, \(P\) must also be false.
    \item Thus, we have shown that “If not \(Q\), then not \(P\)” is true, which by the equivalence of implication, means “If \(P\), then \(Q\)” is true.
\end{enumerate}

\begin{example}
Suppose \( x \in \mathbb{Z} \). If \( x^2 - 6x + 5 \) is even, then \( x \) is odd.
\end{example}

\begin{proof}
The original proposition is equivalent to that if \( x \) is not odd, then \( x^2 - 6x + 5 \) is odd. 

Thus, when \( x \) is even,  \( x = 2a \) for some integer \( a \). So, $$x^2 - 6x + 5 = (2a)^2 - 6(2a) + 5$$  $$= 4a^2 - 12a + 5 = 4a^2 - 12a + 4 + 1$$ $$= 2(2a^2 - 6a + 2) + 1$$ Therefore \( x^2 - 6x + 5 = 2b + 1 \), where \( b \) is the integer \( 2a^2 - 6a + 2 \). Consequently \( x^2 - 6x + 5 \) is odd. Therefore \( x^2 - 6x + 5 \) is not even.

Hence, for \( x \in \mathbb{Z} \), if \( x^2 - 6x + 5 \) is even, then \( x \) is odd.
\end{proof}
\begin{remark}
    In short, proving by contrapositive is just a way to prove a given proposition in a more approachable way through its logical equivalence.
\end{remark}
\subsection{Exercises}
\begin{exercise}
Prove that for any two integers \( a \) and \( b \), if \( a \cdot b \) is odd, then both \( a \) and \( b \) are odd.
\end{exercise}

\begin{proof}
We will prove this by contrapositive. The contrapositive of the given statement is: If either \( a \) or \( b \) is not odd (that is, at least one of them is even), then \( a \cdot b \) is not odd (that is, \( a \cdot b \) is even).

Assume that at least one of the integers, without loss of generality say \( a \), is even. Then \( a \) can be written as \( a = 2k \) for some integer \( k \). The product \( a \cdot b \) is then \( (2k) \cdot b = 2(k \cdot b) \). Since \( k \cdot b \) is an integer, \( 2(k \cdot b) \) is clearly even. Hence, the product \( a \cdot b \) is even.

Since we have shown that the contrapositive statement is true, the original statement must also be true. Therefore, if \( a \cdot b \) is odd, then both \( a \) and \( b \) must be odd.
\end{proof}
\begin{remark}
    For contrapositive of the original statement, the negation of "both" will be  "either", and the negation for "all" is naturally "not all"  
\end{remark}

\begin{exercise}
Prove that if \( n \) is an integer and \( 3n + 2 \) is even, then \( n \) is even.
\end{exercise}

\begin{proof}
Let us prove the statement by contrapositive. Assume \( n \) is not even, which means \( n \) is odd. By definition, an odd number can be written as \( n = 2k + 1 \) for some integer \( k \). We can then express \( 3n + 2 \) as follows:
\begin{align*}
3n + 2 &= 3(2k + 1) + 2 \\
&= 6k + 3 + 2 \\
&= 6k + 5 \\
&= 2(3k + 2) + 1.
\end{align*}
Since \( 3k + 2 \) is an integer, \( 2(3k + 2) \) is even, and adding 1 to an even number results in an odd number, it follows that \( 3n + 2 \) is odd.

Thus, we have shown that if \( n \) is not even, then \( 3n + 2 \) is not even. By contrapositive, this means that if \( 3n + 2 \) is even, then \( n \) must be even.
\end{proof}

\begin{exercise}
For any integers \( a \) and \( b \), the condition \( a + b \geq 15 \) implies that \( a \geq 8 \) or \( b \geq 8 \).
\end{exercise}
\begin{remark}
    Note that the negation of the conclusion in the original claim requires changing the logical "or" to an "and".
\end{remark}
\begin{proof}
The contrapositive of the given claim states that if both \( a \) and \( b \) are integers less than 8, then their sum \( a + b \) is less than 15.

Assume that \( a \) and \( b \) are such integers with \( a < 8 \) and \( b < 8 \). Being integers, the greatest values they can take are \( a = 7 \) and \( b = 7 \). Summing these maximal values yields \( a + b = 7 + 7 = 14 \), which is less than 15.

Thus, having \( a + b < 15 \) necessarily implies that both \( a \) and \( b \) must be less than 8, which completes the proof by contrapositive. Consequently, if \( a + b \geq 15 \), then it must be that either \( a \geq 8 \) or \( b \geq 8 \).


\end{proof}
%--------------------------------------------------
\section{Mathematical Induction}

Many mathematical problems involve only integers; computers perform operations in terms of integer arithmetic. The natural numbers enable us to solve problems by working one step at a time. After giving a definition of the natural numbers as a subset of the real numbers, we study the principle of mathematical induction. We use this fundamental technique of proof to solve problems such as the following.

\begin{problem}
The Checkerboard Problem. Counting squares of sizes one-by-one through eighty-by-eighty, an ordinary eight-by-eight checkerboard has 204 squares. How can we obtain a formula for the number of squares of all sizes on an \( n \)-by-\( n \) checkerboard?
\end{problem}

\begin{problem}
The Handshake Problem. Consider \( n \) married couples at a party. Suppose that no person shakes hands with his or her spouse, and the \( 2n - 1 \) people other than the host shake hands with different numbers of people. With how many people does the hostess shake hands?
\end{problem}

\begin{problem}
Sums of Consecutive Integers. Which natural numbers are sums of consecutive smaller natural numbers? For example, 30 = 9 + 10 + 11 and 31 = 15 + 16, but 32 has no such representation.
\end{problem}

\begin{problem}
The Coin-Removal Problem. Suppose that \( n \) coins are arranged in a row. We remove heads-up coins, one by one. Each time we remove a coin we must flip the coins still present in the (at most) two positions surrounding it. For which arrangements of heads and tails can we remove all the coins? For example, \( \text{THTHT} \) fails, but \( \text{THHHT} \) succeeds. Using dots to denote gaps due to removed coins, we remove \( \text{THHHT} \) via \( \text{THHT} \), \( \text{.H.T} \), \( \text{..HT} \), \( \text{...H} \), \( \text{....} \).
\end{problem}

It is noticeable that these problems share something in common, which is actually the scale of the problem. Also, these problems provide us some procedures that is usable in other cases, making it possible to expand our confirmatory conclusion from the minimum scale to the infinity, or rather, all cases. 

\subsection{Framework of MI}
The base of MI is what we call \textbf{Principle of Induction}.
\begin{theorem}[Principle of Induction]
For each natural number \( n \), let \( P(n) \) be a mathematical statement. If properties (a) and (b) below hold, then for each \( n \in \mathbb{N} \) the statement \( P(n) \) is true.
\begin{enumerate}
    \item[\textbf{a)}] \( P(1) \) is true.
    \item[\textbf{b)}] For \( k \in \mathbb{N} \), if \( P(k) \) is true, then \( P(k + 1) \) is true.
\end{enumerate}
\end{theorem}

\begin{proof}
Let \( S = \{n \in \mathbb{N} : P(n) \text{ is true}\} \). By definition, \( S \subseteq \mathbb{N} \). On the other hand, (a) and (b) here imply that \( S \) satisfies (a) and (b) of Definition 3.5. Since \( \mathbb{N} \) is the smallest such set, \( \mathbb{N} \subseteq S \). Therefore \( S = \mathbb{N} \), and \( P(n) \) is true for each \( n \in \mathbb{N} \).
\end{proof}
\begin{remark}
    Set may be not yet a concept known to you as we only discuss this topic in later chapters. You may ignore this piece of proof for now if this idea is not known for you at this moment.
\end{remark}

When we proceed to prove something with MI, it follows this pattern:
\begin{proof}
The proof is by mathematical induction.

\textbf{Base Case:} 

First we prove that \( P(n_0) \) holds.

\textbf{Inductive Assumption/hypothesis:}

Then, assume that for \( k \geq n_0 \), \( P(k) \) holds.

\textbf{Inductive Step:} 

Finally, with is assumption, prove \( P(k+1) \) also holds.

By the principle of mathematical induction, \( P(n) \) is true for all integers \( n \geq n_0 \).
\end{proof}

To illustrate, we use the sum of the first nth positive integer as an example.
\begin{example}
Prove, using mathematical induction:
\begin{theorem}
The sum of the first \( n \) positive integers is:
\[
1 + 2 + 3 + \ldots + n = \frac{n(n + 1)}{2}
\]
for all integers \( n \geq 1 \).
\end{theorem}
\end{example}

\begin{proof}
\textbf{Base Case:} For \( n = 1 \), \( 1 = \frac{1(1 + 1)}{2} = 1 \), which is true.

\textbf{Inductive Hypothesis:} Assume the statement is true for \( n = k \), i.e., 
\[
1 + 2 + \ldots + k = \frac{k(k + 1)}{2}
\]

\textbf{Inductive Step:} For \( n = k + 1 \), we have:
\begin{align*}
1 + 2 + \ldots + k + (k + 1) &= \frac{k(k + 1)}{2} + (k + 1) \\
&= \frac{k(k + 1) + 2(k + 1)}{2} \\
&= \frac{(k + 1)(k + 2)}{2}
\end{align*}
which is exactly \( \frac{(k + 1)((k + 1) + 1)}{2} \), and the proof is complete.
\end{proof}

\begin{remark}
    Actually, the example only show part of the MI, as in the real practice, we need to find the statement we would like to prove either by examine or reasonable postulation. Do try to get this idea by completing the problem set for this section.
\end{remark}

\subsection{Strong Mathematical Induction}
Strong mathematical induction is the other method of proving that a statement holds for all natural numbers greater than or equal to some initial value. The main difference between the two methods is that strong mathematical induction allows for the use of the statement for all natural numbers less than the current value $n$ in the inductive step, while mathematical induction only allows for the use of the statement for the previous value $n-1$. This additional flexibility can make strong mathematical induction a more powerful tool for proving certain types of statements.
When we proceed to prove something with MI, it follows this pattern:
\begin{proof}
The proof is by mathematical induction.

\textbf{Base Case:} 

First we prove that \( P(n_0) \) holds.

\textbf{Inductive Assumption/hypothesis:}

Then,  assume that the statement holds for all values $k$ such that $n_0 \leq k < n$

\textbf{Inductive Step:} 

Finally, with is assumption, prove \( P(n) \) also holds.

By the principle of mathematical induction, \( P(n) \) is true for all integers \( n \geq n_0 \).
\end{proof}
Here is an example:
\begin{example}
Prove the following theorem:
    \begin{theorem}
         For all natural numbers $n$, the sum of the first $n$ odd numbers is equal to $n^2$.
    \end{theorem}
\end{example}
\begin{proof}
\textbf{Base Case:}
   - When $n=1$, the sum of the first $n$ odd numbers is $1$. Also, $1^2=1$. Therefore, the statement holds for $n=1$.

\textbf{Inductive Hypothesis:}
   - Assume that $P(k)$ is true for some integer $k\ge 1$. That is, assume that the sum of the first $k$ odd numbers is equal to $k^2$.

\textbf{Inductive Step:}
   - We need to prove that if $P(k)$ is true, then $P(k+1)$ is also true. That is, we need to show that the sum of the first $k+1$ odd numbers is equal to $(k+1)^2$.

   - The sum of the first $k+1$ odd numbers is given by:
     $$1 + 3 + 5 + \cdots + (2k+1) + (2k+3) = \sum_{i=1}^{k+1} (2i-1)$$
     
   - Using the inductive hypothesis, we know that the sum of the first $k$ odd numbers is $k^2$. Therefore, we have:
     $$\sum_{i=1}^{k+1} (2i-1) = k^2 + (2k+1) = (k+1)^2$$
     
   - This shows that if $P(k)$ is true, then $P(k+1)$ is also true.

Therefore, by the principle of mathematical induction, we can conclude that for all natural numbers $n$, the sum of the first $n$ odd numbers is equal to $n^2$.
\end{proof}
\begin{remark}
    Sigma notation is a concise and powerful way to represent summation in mathematics. It is denoted by the Greek letter sigma (\(\Sigma\)). The general form of sigma notation is:

\[\sum_{i=m}^{n} a_i\]

This represents the sum of the terms \(a_i\) for \(i\) ranging from \(m\) to \(n\). Here, \(i\) is the index of summation; \(m\) is the lower limit of summation, where the summation starts; and \(n\) is the upper limit, where the summation ends. Each term \(a_i\) in the series is generated by substituting values of \(i\) from \(m\) to \(n\). For example, the sum of the first \(n\) natural numbers can be expressed as:

\[\sum_{i=1}^{n} i = 1 + 2 + 3 + \cdots + n\]

Sigma notation is particularly useful in dealing with series and sequences in mathematics, allowing complex sums to be written in a more compact and readable form. We will explore its property in detail when we get to know sequence and series later in this book.
\end{remark}
%------------------------------------------------
\subsection{Exercises}
\begin{exercise}
Prove, using mathematical induction, the following theorem.
\begin{theorem}[Sum of the first $n$ positive integers]
    The sum of the squares of the first \( n \) positive integers is:
\[
1^2 + 2^2 + 3^2 + \ldots + n^2 = \frac{n(n + 1)(2n + 1)}{6}
\]
for all integers \( n \geq 1 \).
\end{theorem}
Hint: Follow the same procedure of the example. This is a pure algebraic proof.
\end{exercise}

\begin{proof}
\textbf{Base Case:} For \( n = 1 \), \( 1^2 = \frac{1(1 + 1)(2 \cdot 1 + 1)}{6} = 1 \), which is true.

\textbf{Inductive Hypothesis:} Assume the statement is true for \( n = k \), i.e., 
\[
1^2 + 2^2 + \ldots + k^2 = \frac{k(k + 1)(2k + 1)}{6}
\]

\textbf{Inductive Step:} For \( n = k + 1 \), we have:
\begin{align*}
1^2 + 2^2 + \ldots + k^2 + (k + 1)^2 &= \frac{k(k + 1)(2k + 1)}{6} + (k + 1)^2 \\
&= \frac{k(k + 1)(2k + 1) + 6(k + 1)^2}{6} \\
&= \frac{(k + 1)(k(2k + 1) + 6(k + 1))}{6} \\
&= \frac{(k + 1)(2k^2 + 7k + 6)}{6} \\
&= \frac{(k + 1)(k + 2)(2k + 3)}{6}
\end{align*}
which completes the proof.
\end{proof}

\begin{exercise}
    For \( n \in \mathbb{N} \), prove that
\[
\sum_{i=1}^{n} (2i - 1)^2 = \frac{n(2n-1)(2n+1)}{3}.
\]
\end{exercise}
\begin{proof}
    We aim to prove that for all \( n \in \mathbb{N} \), the following formula holds:
\[
\sum_{i=1}^{n} (2i - 1)^2 = \frac{n(2n-1)(2n+1)}{3}.
\]

\textbf{Base Case:}
Let \( n = 1 \).
\[
\sum_{i=1}^{1} (2i - 1)^2 = (2 \cdot 1 - 1)^2 = 1^2 = 1,
\]
and
\[
\frac{1(2 \cdot 1 - 1)(2 \cdot 1 + 1)}{3} = \frac{1 \cdot 1 \cdot 3}{3} = 1.
\]
Since both sides equal 1, the base case holds.

\textbf{Inductive Hypothesis:}
Assume the statement is true for some positive integer \( k \). That is,
\[
\sum_{i=1}^{k} (2i - 1)^2 = \frac{k(2k-1)(2k+1)}{3}.
\]

\textbf{Inductive Step:}
We must show that the statement holds for \( k+1 \). Consider the sum up to \( k+1 \):
\[
\sum_{i=1}^{k+1} (2i - 1)^2 = \sum_{i=1}^{k} (2i - 1)^2 + (2(k+1) - 1)^2.
\]
Using the inductive hypothesis, we can write this as:
\[
\frac{k(2k-1)(2k+1)}{3} + (2k+1)^2.
\]
Simplifying the right-hand side, we get:
\[
\frac{k(2k-1)(2k+1) + 3(2k+1)^2}{3} = \frac{(2k+1)[k(2k-1) + 3(2k+1)]}{3}.
\]
Expanding the terms inside the brackets gives us:
\[
\frac{(2k+1)(2k^2 - k + 6k + 3)}{3} = \frac{(2k+1)(2k^2 + 5k + 3)}{3}.
\]
This simplifies to:
\[
\frac{(2k+1)(2k+3)(k+1)}{3}.
\]
Notice that \( (2k+3) \) is just \( 2(k+1)+1 \), so our expression is equivalent to:
\[
\frac{(k+1)(2(k+1)-1)(2(k+1)+1)}{3},
\]
which matches the right-hand side of our original equation for \( n = k+1 \).

Therefore, by the principle of mathematical induction, the given formula is true for all \( n \in \mathbb{N} \).
\end{proof}

\begin{exercise}
    prove that for all integers $n \geq 1$:
\[
\frac{1}{1 \cdot 3} + \frac{1}{3 \cdot 5} + \frac{1}{5 \cdot 7} + \dots + \frac{1}{(2n - 1)(2n + 1)} = \frac{n}{2n + 1}
\]
\end{exercise}
\begin{proof}
    \textbf{Base Case:} For $n=1$,
    \begin{equation*}
    \frac{1}{1\cdot 3} = \frac{1}{3} = \frac{1}{2 \cdot 1 + 1}
    \end{equation*}
    
    \textbf{Inductive Step:} Assume the statement is true for $n=k$, that is:
    \begin{equation*}
    \sum_{i=1}^{k} \frac{1}{(2i-1)(2i+1)} = \frac{k}{2k+1}
    \end{equation*}
    
    We need to show it holds for $n=k+1$:
    \begin{align*}
    \sum_{i=1}^{k+1} \frac{1}{(2i-1)(2i+1)} &= \sum_{i=1}^{k} \frac{1}{(2i-1)(2i+1)} + \frac{1}{(2(k+1)-1)(2(k+1)+1)} \\
    &= \frac{k}{2k+1} + \frac{1}{(2k+1)(2k+3)} \\
    &= \frac{k(2k+3) + 1}{(2k+1)(2k+3)} \\
    &= \frac{2k^2 + 3k + 1}{(2k+1)(2k+3)} \\
    &= \frac{(k+1)(2k+1)}{(2k+1)(2k+3)} \\
    &= \frac{k+1}{2k+3} \\
    &= \frac{k+1}{2(k+1)+1}
    \end{align*}
    
    Therefore, by the principle of mathematical induction, the statement is true for all integers $n \geq 1$.
    
\end{proof}

\begin{exercise}
    prove that for a fixed nonnegative integer \( q \) and for all positive integers \( n \), the following equation holds:

\[ \sum_{j=1}^{n} j(j+1)(j+2)\ldots(j+q) = \frac{n(n+1)(n+2)\ldots(n+q)(n+q+1)}{q+2} \]
\end{exercise}
Hint: Find the correlation between the sum of the first $n$ and the first $n+1$ term.
\begin{proof}
\textbf{Base Case:}
For \( n = 1 \), the sum on the left-hand side is simply \( 1 \cdot 2 \cdot 3 \ldots (1+q) \), which is the product of the first \( q+1 \) positive integers after 1. The right-hand side is

\[ \frac{1 \cdot 2 \cdot 3 \ldots (1+q)(1+q+1)}{q+2} \]

which, after cancellation of the similar terms in the numerator and the denominator, also equals the product of the first \( q+1 \) positive integers after 1. Hence, the statement holds for \( n = 1 \).

\textbf{Inductive Hypothesis:}
Assume the statement is true for \( n = k \). That is,

\[ \sum_{j=1}^{k} j(j+1)(j+2)\ldots(j+q) = \frac{k(k+1)(k+2)\ldots(k+q)(k+q+1)}{q+2} \]

\textbf{Inductive Step:}
Now we need to show that the statement holds for \( n = k+1 \). Consider the sum

\[ \sum_{j=1}^{k+1} j(j+1)(j+2)\ldots(j+q) \]

This sum can be split into two parts: the sum up to \( k \) (which we assume is correct by the inductive hypothesis) and the \( (k+1)^{st} \) term. So, we have:

$$\sum_{j=1}^{k+1} j(j+1)(j+2)\ldots(j+q) =$$
$$\left( \sum_{j=1}^{k} j(j+1)(j+2)\ldots(j+q) \right) + (k+1)(k+2)(k+3)\ldots(k+q+1) =$$ 
$$\frac{k(k+1)(k+2)\ldots(k+q)(k+q+1)}{q+2} + (k+1)(k+2)(k+3)\ldots(k+q+1)$$



To simplify the right-hand side, factor out the common term \( (k+1)(k+2)(k+3)\ldots(k+q+1) \) from both parts:

\begin{align*}
&= (k+1)(k+2)(k+3)\ldots(k+q+1) \left( \frac{k}{q+2} + 1 \right) \\
&= (k+1)(k+2)(k+3)\ldots(k+q+1) \left( \frac{k + q + 2}{q+2} \right) \\
&= \frac{(k+1)(k+2)(k+3)\ldots(k+q+1)(k+q+2)}{q+2}
\end{align*}

Which is exactly the right-hand side of the original equation for \( n = k+1 \). Therefore, the inductive step is proven.

\end{proof}

\begin{exercise}
    Prove by Mathematical Induction (MI) that for all Natural Number \( n  \),

\begin{enumerate}
    \item[(a)]
    \[ \sum_{j=0}^{n} (j + 1)2^j = n2^{n+1} + 1. \]
    \item[(b)]
    \[ \sum_{j=0}^{n} (j + 1)3^j = \frac{[2n + 1]3^{n+1} + 1}{4}. \]
    \item[(c)]
    \[ \sum_{j=0}^{n} (j + 1)r^j = \frac{[(r - 1)n + (r - 2)]r^{n+1} + 1}{(r - 1)^2} \quad \text{for all numbers } r \neq 1. \]
\end{enumerate}
\end{exercise}
Hint: Examine the expressions. Do we really need 3 proofs?
\begin{proof}
    We will prove (c) by mathematical induction. This also is also a proof of (a) where \( r = 2 \) and (b) where \( r = 3 \). Here \( a = 0 \) and \( P(n) \) is an equation with a LHS and a RHS.

\textbf{Base Case:} If \( n = 0 \) then LHS \( = (0 + 1)r^0 = 1 \),

and RHS \( = \frac{[(r - 1)0 + (r - 2)]r^{0+1} + 1}{(r - 1)^2} = \frac{(r - 2)r + 1}{(r - 1)^2} = 1 \). \( P(1) \) is True.

\textbf{Inductive Hypothesis:} Assume \( \exists k \in \mathbb{N} \) where \( P(k) \) is True.

\textbf{Inductive Step:} If \( n = k + 1 \) then in the predicate \( P \)

\begin{align*}
\text{LHS} &= \sum_{j=0}^{k+1} (j + 1)r^j = \sum_{j=0}^{k} (j + 1)r^j + (k + 1 + 1)r^{k+1} \\
&= \frac{[(r - 1)k + (r - 2)]r^{k+1} + 1}{(r - 1)^2} + (k + 1)r^{k+1} \quad \text{// by Step 2} \\
&= \frac{[(r - 1)k + (r - 2)]r^{k+1} + 1 + (k + 1)(r - 1)^2 r^{k+1}}{(r - 1)^2} \\
&= \frac{[(r(k + 1) - k + (r + 2)] + (k + 2)r - 2r + 1]r^{k+1} + 1}{(r - 1)^2} \\
&= \frac{[(k + 1) + (k + 2)r - (2k + 4)r]r^{k+1} + 1}{(r - 1)^2} \\
&= \frac{[(k + 1)r + (k + 2)(k + 2)r^2 - (k + 2)2r + (k + 2)3]r^{k+1} + 1}{(r - 1)^2} \\
&= \frac{[(k + 1) + (k + 2)r - (2k + 4)r]r^{k+1} + 1}{(r - 1)^2} \\
&= \frac{[(k + 1)r + r - k - 1 - 2]r^{k+1} + 1}{(r - 1)^2} \\
&= \frac{[(r - 1)(k + 1) + (r - 2)]r^{k+1} + 1}{(r - 1)^2} \\
&= \text{RHS} 
\end{align*}

Therefore, \( \forall n \in \mathbb{N}, \sum_{j=0}^{n} (j + 1)r^j = \frac{[(r - 1)n + (r - 2)]r^{n+1} + 1}{(r - 1)^2} \).

Therefore, when \( r = 2, \forall n \in \mathbb{N}, \sum_{j=0}^{n} (j + 1)2^j = \frac{[(2 - 1)n + (2 - 2)]2^{n+1} + 1}{(2 - 1)^2} = n2^{n+1} + 1 \)

and when \( r = 3, \forall n \in \mathbb{N}, \sum_{j=0}^{n} (j + 1)3^j = \frac{[(3 - 1)n + (3 - 2)]3^{n+1} + 1}{(3 - 1)^2} = \frac{[2n + 1]3^{n+1} + 1}{4} \).
\end{proof}

\begin{exercise}
Suppose that $a(0) = 4$, $a(1) = 6$ and $a(n+1)=2a(n)$ - a(n-1) for $n > = 1$.
By writing down a few terms of this sequence, suggest a (non-recursive) formula for $a(n)$ and prove that your formula is correct using strong induction.
\end{exercise}
\begin{proof}[Solution]
	First we calculate the first few terms of the sequence:
	\begin{align*}
		a(2) &= 2a(1) - a(0) = 2 \cdot 6 - 4 = 8, \\
		a(3) &= 2a(2) - a(1) = 2 \cdot 8 - 6 = 10, \\
		a(4) &= 2a(3) - a(2) = 2 \cdot 10 - 8 = 12, \\
		&\vdots
	\end{align*}
	It appears that \( a(n) = 2n + 4 \). Now we prove by strong induction.
	
	\textbf{Base case:} \( a(0) = 4 \) and \( a(1) = 6 \) satisfy the formula \( a(n) = 2n + 4 \).
	
	\textbf{Inductive step:} Assume \( a(k) = 2k + 4 \) holds for all \( k \leq n \). Then for \( n+1 \),
	\begin{align*}
		a(n+1) &= 2a(n) - a(n-1) \\
		&= 2(2n + 4) - (2(n-1) + 4) \\
		&= 4n + 8 - 2n + 2 - 4 \\
		&= 2n + 6 \\
		&= 2(n+1) + 4,
	\end{align*}
	which is the desired formula. Therefore, by strong induction, the formula holds for all \( n \geq 0 \).
\end{proof}


%------------------------------------------------